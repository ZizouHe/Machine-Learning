\documentclass[twoside,11pt]{article}

% Any additional packages needed should be included after jmlr2e.
% Note that jmlr2e.sty includes epsfig, amssymb, natbib and graphicx,
% and defines many common macros, such as 'proof' and 'example'.
%
% It also sets the bibliographystyle to plainnat; for more information on
% natbib citation styles, see the natbib documentation, a copy of which
% is archived at http://www.jmlr.org/format/natbib.pdf

\usepackage{jmlr2e}

% Definitions of handy macros can go here

\newcommand{\dataset}{{\cal D}}
\newcommand{\fracpartial}[2]{\frac{\partial #1}{\partial  #2}}

% Heading arguments are {volume}{year}{pages}{date submitted}{date published}{paper id}{author-full-names}


% Short headings should be running head and authors last names

\firstpageno{1}

\begin{document}

\title{Recursive Teaching Dimension Versus VC Dimension}

\author{\name Zehua Lai\\
       \addr 2014012668
       \AND
       \name Ziheng Huang\\
       \addr 2014012678
       \AND
       \name He Li\\
       \addr 2014012684}

\maketitle

\begin{abstract}
In this work we study the quantitative relation between the recursive teaching dimension
(RTD) and the VC dimension (VCD) of concept classes of finite sizes. The RTD of a concept
class is a combinatorial complexity measure characterized
by the worst-case number of examples necessary to identify a concept according
to the recursive teaching model. An open problem is whether any
linear function of the VCD of a finite concept class gives an upper bound on the recursive teaching
dimension \citep{Zilles2013Models} of that class. In this article, we will review related works in this topic and discuss about the challenges in fully solving the open problem.
\end{abstract}

\section{Introduction}

Sample complexity is one of the central concepts in machine learning. It is
the amount of data needed to achieve a desired learning accuracy. For example, in PAC-learning, sample complexity is characterized
by the VC dimension (VCD) of the concept class \citep{Vapnik1971On}. PAC-learning is a passive learning model. In this model, the role of the teacher is limited
to providing labels to data randomly drawn from the underlying distribution. Different from PAC-learning, there are important models in which teacher involves more actively
in the learning process. For example, in the classical teaching model \citep{goldman1995complexity}, the teacher chooses a set of labeled examples so that the learner,
after receiving the examples, can distinguish the target concept from all other concepts in the
concept class. In this model, the key complexity measure of a concept class is the teaching dimension,
which is defined as the worst-case number of examples needed to be selected by the teacher.

One of the most important questions in computational learning theory is how such information
complexity parameters relate to each other. As \citet{simon2015open} posed, an open problem is: is the RTD upper-bounded by a function that grows only linearly in the VCD?

\section{Definitions and Notation}

$X$ denotes a finite set�� Sometimes, for brevity, we let $ X = [n] := \{1,2,...,n\}$. A concept $C$ is a subset of $X$. $\mathcal{C}$ denotes a concept class over $X$, which is a subset of $2^{|X|}$. For $X' \subseteq X$, we define $\mathcal{C}_{|X'} := \{C \cap X' | C\in \mathcal{C}\}$. We treat concepts interchangeably
as subsets of $X$ and as 0,1-valued functions on $X$. A labeled example is a pair $(x,l)$ with
$x \in X$ and $l \in \{0, 1\}$. A labeled example $(x,l)$ is consistent with a concept $C$ if $\chi(x \in C) = l$. If $S$ is a set of labeled examples, we define $X(S) = \{x\in X|(x,*) \in S\}$.

$A \subseteq X$ is said to be shattered by $\mathcal{C}$ if $\mathcal{C}_{|A} = 2^{|A|}$. VCD($\mathcal{C}$) denotes the VC-dimension of a concept class $\mathcal{C}$, which is the maximum size of a shattered subset of $X$.

A teaching set for a concept $C \in \mathcal{C}$ is a set $S$ of labeled examples such that $C$, but no
other concept in $\mathcal{C}$, is consistent with $S$. Let $\mathcal{TS}(C;\mathcal{C})$ denote the family of teaching sets for
$C \in \mathcal{C}$, let ${TS}(C;\mathcal{C})$ denote the size of the smallest teaching set for $C \in \mathcal{C}$.

\begin{definition}[\citealt{Zilles2013Models}] A teaching plan $P$ for $\mathcal{C}$ is a sequence $((C_1,S_1),...,(C_N,S_N))$, such that $\mathcal{C}=\{C_1,...,C_n\}$ and $\forall t = 1,2,...,N$, $S_t \in \mathcal{TS}(C_t;\{C_t,...,C_N\})$. The quantity ord($P$) $:= max_{t=1,...,N} |S_t|$ is called the order of the teaching plan $P$. Finally, we define recursive teaching dimension RTD($\mathcal{C}$) $:= min\{ord(P) |$ P is a teaching plan for $\mathcal{C}$\}.
\end{definition}

It is clear that RTD$(\mathcal{C})$ is monotonic. If $\mathcal{C}' \subseteq \mathcal{C}$, RTD$(\mathcal{C'})\leq$ RTD$(\mathcal{C})$. We have several propositions which can be used as equivalent definitions for recursive teaching dimension.

\begin{proposition}
$P$ is called a canonical teaching plan for $\mathcal{C}$, if for any $i, j \in \{1,...,N\}$, $i < j$, we have $TS(C_i;\{C_i,...,C_N\})\leq TS(C_j;\{C_i,...,C_N\})$. For a canonical teaching plan $P^*$, RTD$(\mathcal{C})$ $= ord(P^*)$.
\end{proposition}
\begin{proof}
For a canonical teaching plan $P^*$, it is clear that RTD($\mathcal{C}$) $\leq ord(P^*)$. We need to prove RTD($\mathcal{C}$) $\geq ord(P^*)$. Let $P^* = ((C_1,S_1),...,(C_N,S_N))$, and another teaching plan $P = ((C_1',S_1'),...,(C_N',S_N'))$. Choose the minimal $j$ such that $|S_j|=TS(C_j;\{C_j,...,C_N\})=ord(P^*)$. Choose the minimal $i$ such that $C_i' \in \{C_j,...,C_N\}$. So, $C_i'=C_k$ for some $k$. By the definition of canonical teaching plan, $TS(C_k;\{C_j,...,C_N\})\geq TS(C_j;\{C_j,...,C_N\})\geq ord(P^*)$. $ord(P)\geq TS(C_i';\{C_i',...,C_N'\}) \geq TS(C_k;\{C_j,...,C_N\}) \geq ord(P^*)$ hence finish the proof.
\end{proof}
\begin{proposition}
Let $TS_{min}(\mathcal{C}) := min_{C \in \mathcal{C}} TS(C;\mathcal{C})$. Then RTD$(\mathcal{C})= max_{\mathcal{C}' \subseteq \mathcal{C}} TS_{min}(\mathcal{C}')$
\end{proposition}
\begin{proof}
For a canonical teaching plan $P$, we have $TS_{min}(\mathcal{C}) = TS(C_1;\mathcal{C})$. It follows that RTD$(\mathcal{C})$ $ = max \{TS(C_1; \mathcal{C})$, RTD$(\mathcal{C} \backslash C_1)\}$ $ = max \{TS_{min}(\mathcal{C})$, RTD$(\mathcal{C}\backslash C_1)\}$. RTD$(\mathcal{C})$ $\leq max_{\mathcal{C}' \subseteq \mathcal{C}} TS_{min}(\mathcal{C}')$ follows inductively. As for the reverse direction, let $\mathcal{C}' \subseteq \mathcal{C}$ be the maximizer of $TS_{min}$. We have RTD$(\mathcal{C})\geq$ RTD$(\mathcal{C'})\geq TS_{min}(\mathcal{C}')= max_{\mathcal{C}' \subseteq \mathcal{C}} TS_{min}(\mathcal{C}')$
\end{proof}

The open problem under consideration is that whether RTD can be linear bounded by VCD.

\section{RTD for Special Classes}
The following families of concept classes $\mathcal{C}$ are known to have recursive teaching dimension of size equal or smaller than VCD($\mathcal{C}$).

\begin{definition} A concept class $\mathcal{C}$ of VC-dimension $d$ over $X$ of cardinality $n$ is called maximum classes if $\mathcal{C}$ achieved the Sauer's bound\citep{Sauer1972On}, $|\mathcal{C}|=\sum _{i=0}^d C^i_n$.
\end{definition}

\begin{definition} A concept class $\mathcal{C}$ is called intersection-closed if the intersection of any two concepts from $\mathcal{C}$ is itself a concept
in $\mathcal{C}$ as well.
\end{definition}

\begin{definition} Let $\mathcal{F}$ be a vector space of real-valued functions over some domain $X$ and $h : X \to R$. For every $f \in \mathcal{F}$, let $C_f(x)=1$ if $h(x)+f(x) > 0$, else $C_f(x)=0$. Then $D_{\mathcal{F},h} = \{C_f|f\in \mathcal{F}\}$ is called a Dudley class. The VC-dimension of $D_{\mathcal{F},h}$ is
equal to the dimension of the vector space $\mathcal{F}$.
\end{definition}

In \citet{Darnstdt2016Order}, it has been proved that

\begin{theorem} For a concept class $\mathcal{C}$, if it is a maximum class, intersection-closed class or of VC-dimension 1, then RTD($\mathcal{C}$) = VCD($\mathcal{C}$). If it is a Dudley class, then RTD($\mathcal{C}$) $\leq$ VCD($\mathcal{C}$).
\end{theorem}

Above result involved explicit construction of a teaching plan or some compression schemes. However, for general concept classes, a explicit construction may be hard to find. For the Dudley class case, \citet{Ben1998Combinatorial} prove that Dudley classes of dimension $k$ are embeddable in maximum classes of VC-dimension $k$. It is not known to date whether every concept class C of VC-dimension $d$ can be embedded into a maximum concept class of VC-dimension $O(d)$. Indeed, finding such an embedding is considered as a promising method for solving the open problem. \citet{liu2016teaching} also analysed some special settings and gave the teaching dimension for ridge regression, support vector machines, and logistic regression.

Another worth noting approach for analysing teaching complexity is the algebraic method. \citet{samei2014algebraic} used algebraic method to proved several important propositions of RTD and VCD, including the Sauer's bound for RTD and the properties of VCD and RTD-maximum class. Considering the consistency of this method in proving the Sauer's bound for both RTD and VCD, it is a promising method to find more connections between the two complexity concepts.

\section{Quadratic Upper Bound of RTD}

In the paper of \citet{chen2016recursive}, they proved the first bound for RTD of general concept class. Let $f(d)=max_{\mathcal{C}:VCD(\mathcal{C})\leq d}TS_{min}(\mathcal{C})$. They used the method in \citet{kuhlmann1999teaching} and \citet{moran2015compressing} to prove that $f(d)\leq 2^d(d-1)+1 + f(d-1)$ and $f(d)\leq 2^{d+1}(d-2)+d+4$ by induction.

\citet{hu2017quadratic} generalized this method to analysed $(x,y)$-class. A concept class $\mathcal{C}$ is an $(x,y)$-class if for any $A\subseteq X$, $|A|\leq x$, we have $|\mathcal{C}_{|A}|\leq y$. Following theorem can be proved:

\begin{theorem} For any positive integer $x$, $y$, $z$ such that $y \leq 2x-1$ and $z\leq 2y+1$, the following inequality holds:
$$
f(x+1,z)\leq f(x,y)+\left\lceil \frac{(y+1)(x-1)+1}{2y-z+2}\right\rceil
$$
\end{theorem}

Noting that when $x = d-1, y=2^{d-1}-1, z=2^d-1$, the inequality coincides with $f(d)\leq 2^d(d-1)+1 + f(d-1)$, so it is a generalization of the latter inequality. The essential finding is the following: for a concept class with VC-dimension $d$, it it not only a $(d+1,2^{d+1}-1)$-class, but also a $(n,\sum _{i=0}^d C^i_n)$-class. Furthermore, to optimize the induction steps from $f(1,1)$ to $f(x,y)$, we can show that a exponential growth is the best choice by simple calculation. $f(x,\lfloor a^x \rfloor)$ where $a=1.71757$ and $x=4.71607d$ will give us the quadratic bound. In fact, the exponential bound in \citet{chen2016recursive}'s method has not really been improved, we just found a clever way to ignore it. We may still need a new combinatorial technic to fully solve the problem.

\vskip 0.2in
\bibliography{samplebib}

\end{document}
