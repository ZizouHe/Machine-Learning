\documentclass{article}
\usepackage{ragged2e}
\usepackage{amsmath}
\justifying\let\raggedright\justifying
\author{Huang Ziheng}
\title{Lecture 11}
\begin{document}
\maketitle
\section{Multiplicative Weight Updating}
Sanjeew Arora's paper

pmf $x=(x(1),...,x(n))$ $x\geq0,\sum x_i=1$ where $x$ is unknown to the learner\\
At time t, adversary chooses $f_t\in{0,1}^n$, calculates $<f_t,x>=sum_{i=1}^nf_t(i)x_i \in [0,1]$ releases to the learner\\
Learning alg: parameter $\delta,\epsilon$\\
Initialize $x_0=(1/n,...,1/n)$ For $t=1,...,T$ if $(<f_t,x>-<f_t,x_t>)>\delta$\\
${x_t[i]\leftarrow x_{t-1}[i]}(1+\epsilon)$ for all $i$ s.t. $f_t(i)=1$ else keeps the same. Then normalize $x_t$. Else if $(<f_t,x>-<f_t,x_t>)<-\delta$ symmetric operation\\
Thm: For every choice of $f_1,f_2,...$ the alg goes into these two situations for at most $\frac{2\log{n}}{\epsilon\delta}$ (if $0<\epsilon<\delta$)\\
(Homework 1 to prove it with potential function)\\
\section{K-Cluster}
Criterion: k-cluster, learn $c_1,...c_k$ (cluster center) data: $x_1,...,x_n\in R^d$ and $x_i\xrightarrow{nearest} c_{j(i)}$\\
Objective: $\min\limits_{c_1,,,c_k} \sum_{i=1}^n||x_i-c_{j(i)}||^2$\\
$\textcircled{1}$ Partition\\
$\textcircled{2}$ Compute $c_1,...,c_k$\\
$\textcircled{3}$ Partition $x_i\mapsto c_{j(i)}$\\
Which is k-means\\
k-means++\\
Random Initialization: $c_1,...,c_k$\\
$\phi_{OPT}=\sum_i||x_i-c_{j(i)}||^2$\\
$\phi_{K-means++}=???$ s.t. $E[\phi_{K-means++}]\leq 8(\log{k}+2)\phi_{OPT}$\\
$c_i\sim \frac{D(x)^2}{\sum_{x}D(x)^2}$ where $D(x)=||x-C_{(x)}||$ and $C_{(x)}=argmin{||x-c_j||^2,j<i}$
\section{Reinforcement Learning}
Trial and Error
\paragraph{Markov Decision Process}
Definition(MDP): S(State), P(Probability), A(Action), R(Reward). But actually it should be written as $P_{S,A}^S$, $R_{S,A}$ or Transition prob $P(S_{t+1}|S_t,a_t)$ and Reward $R_{t+1}=E[R(S_t,a_t)]$\\
Goal: Maximize long-term reward.  At every time $t$ reward, $\gamma\in(0,1]$, then $\max G_t:=R_{t+1}+\gamma R_{t+2}+...$\\
Policy $S\mapsto a$ and $\pi:S\mapsto A$, Given policy $\pi$, value function $v_\pi(s)=E[R_{t+1}+\gamma R_{t+1}+...]$\\
Action Value Function: $q_\pi(s,a):=E[R_{t+1}+\gamma R_{t+2}+...|S_t=s,A_t=a]$ at time $t$, take action a, then follow $\pi$ at time $t+1$ and so on.
\paragraph{Bellman Equation}
Given policy $\pi$, $v_\pi(s)=R(s,\pi(s))+\gamma\sum_{s'}P(s'|s,\pi(s))v_\pi(s')$ then by vector and matrix it is $v_\pi=R^{(\pi)}+\gamma P^{(\pi)}v_\pi$\\
Now denote $\phi_\pi(v):=R^{(\pi)}+\gamma P^{(\pi)}v$ which is called Bellman Expectation Operator.\\
Thm For $\gamma\in(0,1)$, define $d(v,v')=\max\limits{s\in S}|v(s)-v'(s)|$ then Bellman Expectation Operator is a Contraction mapping. So by iteration we can simply calculate $v_\pi(s)$ by Bellman Operator.\\
$q_\pi(s,a)=R(s,a)+\gamma\sum_{s'}P(s'|s,a)+q_\pi(s',\pi(s'))$ so it is an evaluation for policy $\pi$
\end{document}
